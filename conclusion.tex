\section{Conclusions}\label{sec:conc}
We have presented the photometricresults from the \flare\, simulations, a suite of zoom simulations run using the \eagle\, \citep{schaye_eagle_2015} simulation model probing a wide range of overdensities in the Epoch of Reionisation ($z\ge5$). The wide range of overdensities sampled from a large periodic volume allows us to probe brighter and more massive galaxies in the EoR. Our main findings are as follows:
\begin{itemize}
	\item The \flares\, UV luminosity function provides excellent match to current observations of high redshift galaxies. The UV LF exhibits a Schechter form at $z=6,7,8$ while prefering a double power-law form at $z=5,9,10$. The number density of bright objects at the knee of the function increases by almost 2 orders of magnitude. At $z>8$ the number density of galaxies at the bright-end as predicted by \flares\, is less than the Schechter function fits from other simulation studies. The normalisation of the UV LF is strongly dependent on the environment, with the shape being affected to a lesser extend.  
	\item The relationship between the UV-continuum slope, $\beta$ and M$_{1500}$ of the \flares\, galaxies are in very good agreement with the observations. We find a flattening of the relation at the bright-end. The attenuation in the far-UV also shows a linear relationship with the observed as well as the intrinsic UV luminosity.
	\item We find good agreement of observed line luminosity and equivalent width relationship of the combined [OIII]$\lambda$4959,5007 and H$\beta$ lines at $z=7,8$. 
\end{itemize}
Future observations from \textit{Webb}, \euclid\, and \textit{Roman Space Telescope} will provide further constrains on the photometric properties of these high redshift galaxies. Complimentary observations in the far-IR by \textit{ALMA} will also be instrumental in providing additional constraints on the nebular emission characteristics. We will also be investigating the emission features from the photo-dissociation regions (PDRs) in a future work. 